\newpage
\vspace*{10pt}
\thispagestyle{empty}

\begin{center} \emph{\begin{large}  About InVesalius \end{large}}
\vspace{2pt}
\end{center}

\onehalfspacing
 		
%InVesalius é um software público para a área de saúde que realiza análise e segmentação de
%modelos anatômicos virtuais, possibilitando a confecção de modelos físicos com o auxílio da
%prototipagem rápida.
%A partir de imagens em duas dimensões (2D) obtidas por meio de equipamentos de Tomografia
%Computadorizada (TC) ou Ressonância Magnética (RM), o programa permite criar modelos
%virtuais em três dimensões (3D) correspondentes às estruturas anatômicas dos pacientes em
%acompanhamento médico.

InVesalius is a public health software that performs analysis and segmentation of
Virtual anatomical models, enabling the creation of physical models with the aid of
rapid prototyping (3D printing).
From two-dimensional (2D) images obtained by means of Tomography Computerized (CT) or Magnetic Resonance (MRI), the program allows to create
three-dimensional (3D) anatomical structures corresponding the patients in medical follow-up.

%O nome InVesalius é uma homenagem ao médico belga Andreas Vesalius (1514-1564),
%considerado o "pai da anatomia moderna".
%O software InVesalius é desenvolvido pelo CTI (Centro de Tecnologia da Informação Renato
%Archer), unidade do Ministério da Ciência e Tecnologia (MCT), desde 2001. Inicialmente, apenas
%o programa de instalação era distribuído gratuitamente. A partir de novembro de 2007,
%o InVesalius foi disponibilizado como software livre no Portal do Software Público
%(\href{http://www.softwarepublico.gov.br}{www.softwarepublico.gov.br}), consolidando comunidades de usuários e de desenvolvedores.
%Trata-se de uma ferramenta simples, livre e gratuita,
%robusta, multiplataforma, com comandos em Português, com funções claras e diretas, de fácil
%manuseio e rápida quando executada em microcomputador PC.

InVesalius name is a tribute to the Belgian doctor Andreas Vesalius (1514-1564),
considered the "father of modern anatomy".
InVesalius software is developed by CTI (Center for Information Technology Center Renato
Archer), a unit of the Brazilian Ministry of Science and Technology (MCT), since 2001. Initially, only
the installation program was distributed as freeware. On the November 2007
InVesalius was made available as free software and open source in the Public Software Portal
(\href{http://www.softwarepublico.gov.br}{www.softwarepublico.gov.br}), consolidating communities of users and developers.
It is a simple, easy to use, robust, cross-platform and free tool.

%O uso das tecnologias de visualização e análise tridimensional de imagens médicas, integradas
%ou não a prototipagem rápida, auxiliam o cirurgião no diagnóstico de patologias e permitem que
%seja realizado um planejamento cirúrgico detalhado, simulando com antecedência intervenções
%complexas, que podem envolver, por exemplo, alto grau de deformidade facial ou a colocação
%de próteses.

The use of visualization technologies and three-dimensional analysis of medical images , perhaps integrated with rapid prototyping, assists the surgeon in diagnosing pathologies and a detailed surgical planning, simulating complex interventions in advance, which may involve, for example, a high degree of facial deformity or of prosthetics.

%O InVesalius tem demonstrado grande versatilidade e vem contribuindo com diversas áreas,
%dentre as quais medicina, odontologia, veterinária, arqueologia e engenharia.

InVesalius has demonstrated great versatility and has contributed to several areas,
including medicine, dentistry, veterinary medicine, archeology and engineering.
		
\noindent
